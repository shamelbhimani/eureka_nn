\documentclass{article}

%packaging
\usepackage[utf8]{inputenc}
\usepackage[explicit]{titlesec}
\usepackage{amsmath, amsfonts, amssymb, amsthm}
\usepackage{braket}
\usepackage[margin=1.0in]{geometry}
\usepackage{bbold}
\usepackage{fancyhdr}
\usepackage{fancyvrb}
\usepackage{graphicx}
\usepackage{float}
\usepackage{longtable}
\usepackage{array}
\usepackage{tcolorbox}
\usepackage{beamerarticle}
\usepackage{upgreek}
\usepackage{diffcoeff}

%formatting
\pagestyle{fancy}
\fancyhead[L]{\leftmark}
\fancyhead[R]{\thepage}

%Block settings
\mode<article>{%
  \setbeamertemplate{block begin}{
    \begin{tcolorbox}[%
      colback=white,
      colframe=red,
      arc=0mm,
      title=\insertblocktitle,
      colbacktitle=white,
      coltitle=black,
      fonttitle=\bfseries,
      detach title,
      before upper={\tcbtitle\par}
    ]
  }

  \setbeamertemplate{block end}{
    \end{tcolorbox}
  }
}

% Theorem, axiom, and remark settings
\theoremstyle{plain}
\newtheorem{myth}{Theorem}[section]
\newtheorem{myprop}[myth]{Proposition}
\newtheorem{mylemma}[myth]{Lemma}

\theoremstyle{definition}
\newtheorem{mydef}{Definition}[section]
\newtheorem{myax}{Axiom}

\theoremstyle{remark}

% Step 1: Create a counter for myex, dependent on myth
\newcounter{myex}[myth]
\renewcommand{\themyex}{\themyth.\arabic{myex}}

% Step 2: Define myex environment manually
\newenvironment{myex}[1][]{
  \refstepcounter{myex}%
  \par\medskip
  \noindent\textbf{Example \themyex.} #1\par
  \noindent
}{\medskip}



\title{Vanilla Neural Networks}
\author{Shamel Bhimani}
\date{August 2025}

\begin{document}

\maketitle

\tableofcontents
\newpage


\section{Precursors to Backpropagation}\label{sec:precursors-to-backpropagation}
test

  \subsection{Chain Rule}\label{subsec:chain-rule}

  \subsubsection{Core Principle of the Chain Rule}
\noindent The chain rule is a formula for finding the \textit{derivative} of
a \textbf{composite function}, that is, a function that is formed by the composition of two or more other functions.
  It is an indispensable tool in science, engineering, and statistics,
particularly for optimization problems where functions are dependent on a
chain of intermediate variables. In the context of neural networks, the
backpropagation algorithm is, at its core, a highly efficient and systematic
application of the chain rule.

  \subsubsection{The Single-Variable Chain Rule}
  \noindent Let $y$ be a function of $u$, denoted as $y = f(u)$, and let $u$
in turn be a function of $x$, denoted as $u = g(x)$. The composition of
these two functions forms a new function, $y = F(x) = f(g(x))$.\\

  \noindent The \textbf{Chain Rule Theorem} states that if both $f$ and $g$
are differentiable functions, then the derivative of the composition function $F(x)$ with respect to $x$ is the product of the derivative of the outer function $f$ with respect to its input $u$, and the derivative of the inner function $g$ with respect to its input $x$.\\

  \begin{myth}[\textbf{Chain Rule}]
    \label{Chain Rule}
    If $y. = f(u)$ and $u = g(x)$ are differentiable functions, then the
    derivative of $y$ with respect to $x$ is given by:

              \[\frac{dy}{dx} = \frac{dy}{du} \cdot \frac{du}{dx}\]

    An equivalent notation, often used for simplicity, is the prime notation:

              \[F^\prime(x) = f^\prime(g(x)) \cdot g^\prime(x)\]
  \end{myth}
  \medskip
  \begin{myex}
    Let $y=(x^2 + 1)^3$. We can identify this as a composite function.
    Let the inner function be $u = g(x) = x^2 + 1$.
    Let the outer function be $y= f(u) = u^3$.

    First, we find the individual derivatives:
    \begin{gather*}
        \frac{dy}{du} = \frac{d}{du}(u^3) = 3u^2\\
        \frac{du}{dx} = \frac{d}{dx}(x^2 + 1) = 2x\\
    \end{gather*}

    Now, we apply chain rule:

              \[\frac{dy}{dx} = \frac{dy}{du} \cdot \frac{du}{dx} = (3u^2) \cdot (2x)\]

    Finally, we substitute the expression for $u$ back into the result:

              \[\frac{dy}{dx} = 3(x^2 + 1)^2 \cdot (2x) = 6x(x^2 + 1)^2\]
  \end{myex}

  \subsubsection{The Multi-Variable Chain Rule}
  \noindent The concept of the chai rule extends to functions of multiple
variables using partial derivatives. This form is particularly relevant in
fields like optimization and machine learning, where a function's value
depends on a multitude of intermediate parameters.\\

  \noindent Consider a function $z$ that depends on two intermediate
variables $x$ and $y$, such that $z = f(x,y)$. In turn, both $x$ and $y$ are
functions of a third variable, $t$, such that $x = g(t)$ and $y = h(t)$. To
find the total rate of change of $z$ with respect to $t$, we must sum the
contributions from each path of dependency.\\

  \begin{myth}[Multi-Variable Chain Rule]
  If $z = f(x, y)$ is a differentiable function of $x$ and $y$, and $x = g(t)$ and $y = h(t)$ are differentiable functions of $t$, then the total derivative of $z$ with respect to $t$ is:

              \[\diff{z}{t} = \diffp{z}{x}\diff{x}{t} + \diffp{z}{y}\diff{y}{t}\]
  \end{myth}

  \noindent The partial derivative $(\diffp{z}{x^\prime}  \diffp{z}{y})$
account for how $z$ changes with respect to its immediate inputs, while the
derivatives $(\diff{x}{t^\prime}  \diff{y}{t})$ account for how those inputs
change with respect to the ultimate variable, $t$. The sum captures the
total effect.\\

  \noindent \textbf{Generalization:} This rules extends to any number of
intermediate variables. If $z = f(x_1, x_2,\dots,x_n)$, and each $x_i$ is a
function of $t$, then:

            \[\diff{t}{z} = \sum_{i=1}^{n}\diffp{z}{x_i}\diff{x_i}{t} \]


  \begin{myex}
    Let $z = x^2 y^3$, where $x = \sin{t}$ and $y = \cos(t)$. We want to find $\diff{z}{t}$.
    First, we compute the partial derivatives of $z$ with respect to its
    inputs $x$ and $y$:
    \begin{gather}
      \diffp{z}{x} = \diff{}{x}(x^2 y^3) = 2xy^3\\
      \diffp{z}{y} = \diff{}{y}(x^2 y^3) = 3x^2 y^2
    \end{gather}
    Next, we compute the derivatives of the intermediate variables with
    respect to $t$:
    \begin{gather}
      \diff{x}{t} = \diff{}{t}(\sin{t}) = \cos{t}\\
      \diff{y}{t} = \diff{}{t}(\sin{t}) = -\sin{t}
    \end{gather}
    Finally, we apply the multi-variable chain rule formula:
              \begin{gather}
                  \diff{z}{t} = \diffp{z}{x}\diff{x}{t} + \diffp{z}{y}\diff{y}{t}\\
                  \diff{z}{t} = (2xy^3)(\cos{t}) + (3x^2 y^2)(-\sin{t})\\
              \end{gather}
    To obtain the final expression solely in terms of $t$. we substitute $x = \sin{t}$ and $y = \cos{t}$ back into the equation:
            \begin{gather}
              \diff{z}{t} = 2\sin{t}\cos^4{t}-3\sin^3{t}\cos^2{t})
            \end{gather}
  \end{myex}

\end{document}